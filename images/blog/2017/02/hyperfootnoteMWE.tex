\documentclass{article}
\usepackage[parfill]{parskip}    		% Activate to begin paragraphs with an empty line rather than an indent

\usepackage{hyperref} % you need this package

% Make clickable footnote
\newcommand{\hyperfootnote}[1][]{\def\ArgI{{#1}}\hyperfootnoteRelay}
  % relay to new command to make extra optional command possible
\newcommand\hyperfootnoteRelay[2][]{\href{#1#2}{\ArgI}\footnote{\href{#1#2}{#2}}}
  % the first optional argument is now in \ArgI, the second is in #1
  
% Takes at most 3 parameters (see http://www.tex.ac.uk/FAQ-twooptarg.html for info on multiple optional parameters)
% If first parameter isn't given, it's value is '' (empty string in text before footnote reference)
% If second parameter isn't given, it's value is '' (string before visible URL, e.g. 'http://')
% Makes a clickable footnote (alternatively: \url{}) with optional reference in the text as well
% Use 1: \hyperfootnote{www.mywebsite.com}: creates a footnote consisting of a clickable URL
% Use 2: \hyperfootnote[My website]{www.mywebsite.com}: creates a clickable piece of text in the text ('My website') plus a footnote consisting of a clickable URL
% Note: requires the hyperref package.
% Note: use xspace package to add/absorb spaces when necessary (e.g. to avoid a space between the footnote number and a punctuation mark)

% Info on how to define a LaTeX command: https://www.sharelatex.com/learn/Commands


\title{Minimal Working Example of the \texttt{\char`\\ hyperfootnote} command}
	% I used \char`\\ instead of \textbackslah to avoid "Font shape `OMS/cmtt/m/n' undefined" warnings. 
\author{Brecht De Man}
\date{19 February 2017}


\begin{document}

\maketitle % MANDATORY! 


	Have a look at \hyperfootnote[my website][http://]{www.brechtdeman.com}! 

	You will notice that while a short URL is shown in the footnote, both the clickable text in the paragraph and the footnote itself link to the full URL, i.e. including `http://' or `https://', which you add as an optional argument. 

	You can also have non-clickable text\hyperfootnote{https://www.latex-project.org} with a full clickable link as footnote if you prefer. 

	\vspace{2em}
	Syntax:\\
	\texttt{\char`\\ hyperfootnote[<optional text>][<optional invisible URL prefix>]\char`\{URL\char`\}}\\
	E.g. \\
	\texttt{\char`\\ hyperfootnote[my website][http://]\char`\{www.brechtdeman.com\char`\}}\\
	\texttt{\char`\\ hyperfootnote\char`\{https://www.latex-project.org\char`\}}\\

	\vspace{2em}
	Therefore, \\
	\texttt{\char`\\ href\char`\{http://www.brechtdeman.com\char`\}\char`\{my website\char`\}\\
		\char`\\ footnote\char`\{ \char`\\ href \char`\{http://www.brechtdeman.com\char`\} \char`\{www.brechtdeman.com\char`\} \char`\} } \\
	can be replaced by \\
	\texttt{\char`\\ hyperfootnote[my website][http://]\char`\{www.brechtdeman.com\char`\}}\\

\end{document}